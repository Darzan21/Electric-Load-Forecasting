\chapter{Conclusion \& Future Work}
Various models utilizing time-series methods and machine learning techniques for Short Term Load Forecasting have been proposed in the literature, over the past years. There is still an essential need for
the Independent Power Transmission Operator S.A (IPTO) and market participants (energy providers) in electricity markets, to develop accurate and robust forecasting models, that can capture the uncertainties
in the load demand, the seasonal load demand variations and the load demand variations of special days, such as holiday. Furthermore, these models should take care of volatility and non-stationarity issues in the electric load demand,  especially when the whole network is progressing towards smart power systems with a lot of smart grid technology systems already part of modern power grid. As mentioned earlier, the main objective of this work is to provide IPTO with an accurate and convenient short-term load forecasting (STLF) model, in order to increase the power system reliability and reduce the system's OPEX. On the other hand, this model should be used by energy providers for making more accurate forecasting concerning the daily Low Voltage electric load declaration. In the electricity market nowadays, the energy trading system and the spot price establishment are based on accurate load forecasting. Energy providers should optimize this process and constantly improve their models for more accurate forecasts. As seen in the literature, many machine learning, deep learning and statistical modeling approaches exist, concerning STLF.
\par In this research, we proposed a short-term electricity load forecasting model using some state of the art algorithms and using the original hourly load demand data from 1-1-2018 to 31-09-2019, which obtained from the Independent Power Transmission Operator S.A (IPTO). Applying four of them (Time-Series SARIMAX, Prophet, Neural Networks (LSTM) and Light Gradient Boosting Machine-LightGBM Regressor), the forecast mean squared error was smaller in contrast with the OoEM’s rmse, with the best prediction coming from ensemble models and more specifically from LightGBM Regressor. In the last chapters, the forecasting results of our experiments are presented and we observed that the proper selection of the model orders (concerning SARIMAX), the architecture and the hyperparameter tuning (concerning LSTM Neural Network) and the optimal parameter's value choice (concerning LightGBM Regressor) is thought to be crucial for successful electric load demand forecasting. Finally, the proposed model (LightGBM) provides more accurate forecasts concerning Low Voltage hourly load demand. The performance measures evaluated in this thesis are, to the best of the author’s knowledge, offer a good indication of how each one of the proposed algorithms can be optimally utilized.
\par
Time series forecasting in general and especially Short Term Load Forecasting, is a fast growing research area, thus providing a vast majority of alternatives for future work. Moreover, concerning our experiments, in some cases, significant deviation is observed between the actual observations and our predicted values. Thus, we can suggest that a more suitable data preprocessing may improve the forecast performance. Moreover,
other factors affecting electric load demand such as weather data -which was unavailable for the period under study- could be included in the building and improvement of the model. Furthermore, the integration of some economic metrics could be facilitated, as the economic environment has a significant influence on consumer behavior, for example economic growth lead to increased demand for electric load. Last but not least, future work may include actions such as modeling the influence of historical data length on forecast accuracy, studying alternative forecasting algorithms and combining time series forecasts in hybrid models e.g combining different algorithms and models in order to improve forecasting accuracy. More specifically, many techniques proposed in the literature have not been developed in this work and more importantly this thesis still has some future work to do, concerning the problems faced by the entire
STLF area, concerning the learning and training procedures. The results and the forecasting algorithms proposed in this research concerning short term load forecasting should be compared in a realistic
setting before any dependence on the forecasts can be justified.












