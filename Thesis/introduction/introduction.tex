\chapter{Introduction}
\par Nowadays, it is quite clear that the current lifestyle imposes a greater need for electrical energy supply. In our modern urbanized world, population is rapidly increasing and technology is developing in a fast pace leading to an era of industrialization. Such reality imposes a greater need for electrical energy, however, current technological advances do not support its storage from sources like renewables (solar power, wind power etc.), natural gas, coal or nuclear power and therefore it is still not possible to provide a high quality, fast and efficient storage, distribution and supply network. Additionally, studies have shown that the high demand on energy has a significant impact on the world economy. Hence, a lot of research has been conducted on sustainable sources of electricity, ways to balance demand and supply without unfortunate slow-downs as well as stead electricity production. Due to the continuous demand for electricity, the lack of storage and constraints in resources but also the need for immediate electrical consumption the moment it is produced, electricity market actors have started to search and develop action plans and practices for an equilibrium in electrical demand and supply. 
\par Electrical power system basically comprises of a generation, transmission, distribution and utilization system. The major operation and maintenance cost is in the production sector, followed by the distribution sector. In practice, the economic importance of the distribution system is huge and the amount of investment involved, demands extensive planning, design and operation. In this context, planning includes practices aiming at the cost of losses along with the cost minimization of the substations, the sub-transmission networks, the laterals, and so on.  
\par Even though supply and demand play a significant role is this situation, especially in financial terms, there are still other key factors. In detail, the external environment given that electricity generation is depending on external resources, or the legislation and state regulations that lead to limitations on electricity consumption. Having all these in regard, the concept of planning is integrated in activities while producing, transmitting and distributing energy for the prevention of energy shortage or surplus. Hence, it is very important to be able to forecast accurately the load demand, not only for consumers and suppliers but also policymakers so as to plan, regulate and control the operation of electricity power systems.  Thus, load forecasting is an indispensable section of designing, planning and operation of electric utilities, and moreover necessary so as to allocate considerable amount of electric energy in order to increase the economy significantly by electric power manufacturers. In general, electricity load forecasting offers an estimation of the amount of electric load in a given period according to the system data provided. In a broader sense, it seems that numerous areas in power system energy management require efficient and accurate load forecasting models, such as:
\begin{enumerate}
\item \textbf{Distribution System Planning:} Distribution system planning is a very essential task, concerning the growing demand for electrical energy by additions of technically capable and reasonably economical distribution system units. The distribution system planning focuses on ensuring that the increasing load demands along with high load densities are supplied the optimal way by the additional distribution systems. Distribution system planning should also consider the load magnitude and its geographical location. Concerning STLF, efficient and accurate load forecasting is a very critical factor to a cost effective distribution system planning. The system planning decisions greatly depend on the spatial load forecasting that gives the growth pattern of the load pertaining to a time, place and quantity.\\
\\
\\
\item \textbf{Distribution System Expansion:} It seems that the load growth in a particular geographic area served by a particular utility significantly influence the distribution system expansion. Hence, an accurate forecast of the load growth and its effect on the system performance is very important for the expansion process.
\item \textbf{Operation and Maintenance:} A Smart Grid's power system regular operation and maintenance, is highly simplified by an accurate load forecast. It helps in guiding the personnel in making the appropriate switching and loading decisions, pertaining to the load profiles obtained during the load forecasting process.
\item \textbf{Financial Planning:} Accurate electric load forecasting, helps the system planning executives in making appropriate financial decisions related to the expansion of network, approvals of budgets, human resource management, technological upgrades etc.
\item \textbf{Load Management:} The electrical power supply utilities, try to meet all the customer demands for electrical energy whenever that demand occurs. Due to the stringent financial constraints because of the high cost of labor, materials and interest rates, very significant environmental concerns and the ever increasing shortage of fuels, the utilities are seriously diverting towards the option of load management as an alternative to capacity expansion to the extent possible. Effective load forecasting forms a genuine basis for the efficient load management.
\end{enumerate}
